\title{Assignment 1: CS 663, Fall 2023}
\author{Darshan Makwana, Vignesh Nayak, Harsh Kavediya}

\documentclass[11pt]{article}
\usepackage{amsmath}
\usepackage{amssymb}
\usepackage{hyperref}
\usepackage{ulem}
\usepackage[margin=0.5in]{geometry}
\begin{document}
\maketitle

\begin{enumerate}

\item[Q3.] Let $P_{I+J}$ denote the probability mass function for image $I+J$. By definition $P_{I+J}(i)$ denotes the probability that any randomly choosen pixel in the image will have an intensity of $i$
\begin{equation}
    \begin{aligned}
        P_{I+J}(i) &= P_{I+J}(I+J = i) \\
        &= P_{I+J}(I = i - J) \\
        &= P_{I}(I = i - J | J)P_{J}(0 \le j \le i) \\
        &= \sum\limits_{j=0}^{i} P_{I}(i-j)P_{J}(j)
    \end{aligned}
\end{equation}

This resembles the convolution operation we used in class for local spatial filters

\end{enumerate}

\end{document}